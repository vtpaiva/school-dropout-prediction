\section{Conclusão}

\par Portanto, tomando como base de raciocínio o estudo desenvolvido, é razoável concluir que a evasão escolar é um problema que apresenta múltiplas causas, sendo necessária uma visão holística sobre o problema e ações de várias entidades de múltiplos setores para que seja sanada. Além disso, foi possível concluir que as amostras balanceadas, para o propósito de identificar as fragilidades do sistema educacional e para que os modelos apresentem maior sensibilidade para identificação de vulnerabilidades, são mais indicadas para o treinamento de modelos devido ao menor impacto do enviesamento ocasionado por uma classe majoritária.

\par Ademais, fica evidente o poder de generalização do algoritmo de redes neurais que, apesar de apresentar interpretação mais complexa, é capaz de gerar regiões delimitadoras melhores e pode resultar em um modelo mais preciso e uma relevância mais elevada no ensino fundamental para a determinação da taxa de evasão escolar de uma instituição de ensino.

\includemedia[
  label=grafico,
  width=0.9\linewidth,height=0.6\linewidth,
  activate=onclick,
  addresource=grafico.html,
  flashvars={source=grafico.html}
]{}{VPlayer.swf}