\newpage
\section{Introdução}

\par A evasão escolar é um problema significativo que afeta a educação e o desenvolvimento socioeconômico do Brasil. Portanto, compreender os fatores que influenciam a evasão escolar é crucial para desenvolver políticas eficazes que possam reduzir suas taxas e melhorar a qualidade da educação. Desse modo, a partir das bases de dados do censo escolar de 2021 disponibilizadas pelo \textbf{INEP} (Instituto Nacional de Estudos e Pesquisas Educacionais Anísio Teixeira) e dos dados do PIB dos municípios disponibilizados pelo \textbf{IBGE} (Instituto Brasileiro de Geografia e Estatística), tem-se como fito abordar a problemática da evasão escolar por intermédio de técnicas de inteligência artificial.

\par O objetivo deste trabalho é analisar os diversos fatores que têm influência sobre o índice de evasão escolar em todas as escolas do Brasil, utilizando métodos de aprendizado de máquina e mineração de dados. A análise busca identificar fatores que contribuem positivamente, negativamente ou que não possuem influência significativa sobre a evasão escolar. Para alcançar esse objetivo, foram utilizados dois algoritmos de machine learning: Random Forest e Redes Neurais.

\par Ao utilizar essas técnicas, este estudo pretende fornecer uma visão abrangente sobre os fatores que influenciam a evasão escolar, oferecendo subsídios para a formulação de políticas públicas e intervenções educacionais que possam efetivamente combater esse problema. A análise e os resultados obtidos podem servir como uma base sólida para futuras pesquisas e iniciativas voltadas à melhoria do sistema educacional brasileiro.
